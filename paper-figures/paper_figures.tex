

\documentclass[9pt,lineno]{elife}
% Use the onehalfspacing option for 1.5 line spacing
% Use the doublespacing option for 2.0 line spacing
% Please note that these options may affect formatting.
% Additionally, the use of the \newcommand function should be limited.


\usepackage{lipsum} % Required to insert dummy text
\usepackage[version=4]{mhchem}
\usepackage{siunitx}
\usepackage{soul}
\usepackage[colorinlistoftodos]{todonotes}
\usepackage{lipsum}
\usepackage{gensymb}
\DeclareSIUnit\Molar{M}


\title{Title of Dan's paper}

\author[1]{Author Name}
\author[1]{Eve Marder}
\affil[1]{Volen Center and Biology Department, Brandeis University, Waltham MA 02454 USA}



\corr{marder@brandeis.edu}{EM}


\begin{document}

\maketitle

\begin{abstract}
Dan's awesome abstract
\end{abstract}


\section*{Introduction}

\lipsum[2-3]

\section{Results}


\subsection{Single-sensor homeostatic regulation can compensate for changes in cell size}

\lipsum[9]

\begin{figure}
\includegraphics[width=\linewidth]{1.png}
\caption{\textbf{Circuit Diagram + Example traces showing units on each nerve} More caption } 
\label{fig:1}
\end{figure}


\begin{figure}
\begin{fullwidth}
\includegraphics[width=\linewidth]{2.pdf}

\caption{\textbf{Gastric mill rhythms can be evoked at temperatures from 7 to 23 C }  (a) Extracellular recordings at 11\degree C from nerves with gastric units (lgn, dgn, mvn), and nerves with pyloric units (pdn, lvn), immediately after stimulation. Bursting of LG is visible on lgn, bursting of DG on dgn, and PD on pdn. (b) Recordings from the same nerves on the same preparation at 23\degree C. Both gastric and pyloric rhythms speed up at this temperature. (c) Rasters of LG showing evoked bursts that get faster with increasing temperature. (d) LG burst period as a function of time for 8 different preparations. Each line corresponds to stimulation and evoked bursts at a different temperature (see color scale). Vertical lines separate preparations and indicate end of stimulation. } 
\label{fig:2}
\end{fullwidth}
\end{figure}




\subsection{Robustness to perturbations analogous to size change results in sensitivity to some channel-specific perturbations}


\begin{figure}
\begin{fullwidth}
\includegraphics[width=\linewidth]{3.pdf}

\caption{\textbf{Within-preparation comparison of spontaneous and evoked gastric mill rhythms} Let me know if you want me to write out the figure captions, or if you want to do it.  }
\label{fig:2}
\end{fullwidth}
\end{figure}


\lipsum[1-4]

\subsection{Predicting failure of homeostatic compensation}

\begin{figure}[!hbp]
\centering
\begin{fullwidth}
\includegraphics[width=\linewidth]{4.pdf}
\end{fullwidth}
\caption{\textbf{PD and LG bursting speeds up with increasing temperature in a similar manner.} Let me know if you want me to write out the figure captions, or if you want to do it.   }
\end{figure}

\lipsum[3-5]

\subsection{Robustness to scale perturbations persists across projections and neuron models}

\begin{figure}[!hbp]
\centering
\begin{fullwidth}
\includegraphics[width=\linewidth]{5.pdf}
\end{fullwidth}
\caption{\textbf{PD spike rasters triggered by LG burst starts, but not DG, exhibit regularity across temperatures.} Let me know if you want me to write out the figure captions, or if you want to do it.   }
\end{figure}

\lipsum[2-4]

\begin{figure}[!htp]
\centering
\begin{fullwidth}
\includegraphics[width=\linewidth]{6.pdf}
\end{fullwidth}
\caption{\textbf{LG, but not DG, bursting is phase locked to PD across temperature.} (a) Probability of LG burst starts as a function of PD phase. Each curve pools all LG burst starts in a given temperature.  (b) Probability of DG burst starts as a function of PD phase. Each curve in (a) and (b) pools all LG or DG burst starts at a given temperature. Shading indicates standard error.  (c) Mean LG start in PD phase as a function of temperature. Gray lines denote individual preparations; black line is the average of all preparations. Bars indicate standard error of the mean. (d) Probability of LG spiking as a function of PD phase. (e) Proabbility of DG spiking as a function of PD phase.  Dashed lines in all panels indicates chance level.}
\end{figure}


\section{Discussion}




\subsection{Consequences of using calcium concentration as a proxy for cell activity}

\lipsum[5]

\begin{figure}[!htp]
\centering
\begin{fullwidth}
\includegraphics[width=\linewidth]{7.pdf}
\end{fullwidth}
\caption{\textbf{LG, but not DG, bursting is phase locked to PD across temperature.} (a) Probability of LG burst starts as a function of PD phase. Each curve pools all LG burst starts in a given temperature.  (b) Probability of DG burst starts as a function of PD phase. Each curve in (a) and (b) pools all LG or DG burst starts at a given temperature. Shading indicates standard error.  (c) Mean LG start in PD phase as a function of temperature. Gray lines denote individual preparations; black line is the average of all preparations. Bars indicate standard error of the mean. (d) Probability of LG spiking as a function of PD phase. (e) Proabbility of DG spiking as a function of PD phase.  Dashed lines in all panels indicates chance level.}
\end{figure}


\lipsum[1]

\subsection{Subsection title}

\lipsum[2]


\begin{figure}
\begin{fullwidth}
\includegraphics[width=\linewidth]{8.pdf}

\caption{\textbf{Gastric mill rhythms can be evoked at temperatures from 7 to 23 C } Let me know if you want me to write out the figure captions, or if you want to do it.   } 
\label{fig:8}
\end{fullwidth}
\end{figure}


\subsection{Sub section title}

\lipsum[2]


\begin{figure}
\begin{fullwidth}
\includegraphics[width=\linewidth]{9.pdf}

\caption{\textbf{Gastric mill rhythms can be evoked at temperatures from 7 to 23 C } Let me know if you want me to write out the figure captions, or if you want to do it.   } 
\label{fig:9}
\end{fullwidth}
\end{figure}

\subsection{Stochastic effects can arise from low copy numbers}

\lipsum[3]

\subsection{Model predictions for non-neuronal cells}

\lipsum[2]

\subsection{Other mechanisms of compensation for growth}

\lipsum[5-6]




\section{Methods and Materials}

\subsection*{Data analysis}

\subsubsection*{Spike identification and sorting}

\subsubsection*{Measuring burst metrics}

\subsubsection*{Estimating $Q_{10}$s of burst periods}




\appendix

\end{document}
