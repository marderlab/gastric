

\documentclass[9pt,lineno]{elife}
% Use the onehalfspacing option for 1.5 line spacing
% Use the doublespacing option for 2.0 line spacing
% Please note that these options may affect formatting.
% Additionally, the use of the \newcommand function should be limited.


\usepackage{lipsum} % Required to insert dummy text
\usepackage[version=4]{mhchem}
\usepackage{siunitx}
\usepackage{soul}
\usepackage[colorinlistoftodos]{todonotes}
\usepackage{lipsum}
\usepackage{gensymb}
\DeclareSIUnit\Molar{M}


\title{Title of Dan's paper}

\author[1]{Author Name}
\author[1]{Eve Marder}
\affil[1]{Volen Center and Biology Department, Brandeis University, Waltham MA 02454 USA}



\corr{marder@brandeis.edu}{EM}

\begin{document}

\maketitle

\begin{abstract}
Dan's awesome abstract
\end{abstract}


\section*{Introduction}

\lipsum[2-3]

\section{Results}


\subsection{Single-sensor homeostatic regulation can compensate for changes in cell size}

\lipsum[9]

\begin{figure}
\includegraphics[width=\linewidth]{1.png}
\caption{\textbf{Circuit Diagram + Example traces showing units on each nerve} More caption } 
\label{fig:1}
\end{figure}


\begin{figure}
\begin{fullwidth}
\includegraphics[width=\linewidth]{evoked.pdf}

\caption{\textbf{Gastric mill rhythms can be evoked at temperatures from 7 to 23 C }  (a) Extracellular recordings at 11\degree C from nerves with gastric units (lgn, dgn, mvn), and nerves with pyloric units (pdn, lvn), immediately after stimulation. Bursting of LG is visible on lgn, bursting of DG on dgn, and PD on pdn. (b) Recordings from the same nerves on the same preparation at 23\degree C. Both gastric and pyloric rhythms speed up at this temperature. (c) Rasters of LG showing evoked bursts that get faster with increasing temperature. (d) LG burst period as a function of time for 8 different preparations. Each line corresponds to stimulation and evoked bursts at a different temperature (see color scale). Vertical lines separate preparations and indicate end of stimulation. } 
\label{fig:2}
\end{fullwidth}
\end{figure}




\subsection{Robustness to perturbations analogous to size change results in sensitivity to some channel-specific perturbations}


\begin{figure}
\begin{fullwidth}
\includegraphics[width=\linewidth]{within_prep.pdf}

\caption{\textbf{Within-preparation comparison of spontaneous and evoked gastric mill rhythms} (Let me know if you want me to write out the figure captions, or if you want to do it. (a) Comparison of spontaneous (black) and evoked (blue) rhythms in four different preparations. Each row compares extracellular recordings from the same preparation. (b) Inter-spike intervals of LG during spontaneous (black) and evoked (blue) rhythms. (c) Burst periods of LG during spontaneous gastric rhythms (black, ordinate) compared to burst periods of LG during evoked rhythms (blue, abcissa).  }
\label{fig:2}
\end{fullwidth}
\end{figure}


\lipsum[1-4]

\subsection{Predicting failure of homeostatic compensation}

\begin{figure}[!hbp]
\centering
\begin{fullwidth}
\includegraphics[width=\linewidth]{burst_periods.pdf}
\end{fullwidth}
\caption{\textbf{PD and LG bursting speeds up with increasing temperature in a similar manner.} (a) Burst period of  LG (red) and PD (black) as a function of temperature. Each panel shows a different preparation. Large variations in PD burst periods at discrete temperatures follow periods of stimulation. (b) Mean burst periods of LG and DG exponentially with temperature. (c) $Q_{10}$s of PD burst periods compared to $Q_{10}$s of LG burst periods. (d) Firing rate within a burst increases with temperature. In (b) and (d), pale lines show individual preparations while solid lines shows the average of all preparations. }
\end{figure}

\lipsum[3-5]

\subsection{Robustness to scale perturbations persists across projections and neuron models}

\begin{figure}[!hbp]
\centering
\begin{fullwidth}
\includegraphics[width=\linewidth]{PD_triggered_by.pdf}
\end{fullwidth}
\caption{\textbf{LG, but not DG, bursting is phase locked to PD across temperature} (a) Rasters of PD spikes triggered by start of LG bursting (top row) or start of DG bursting (bottom row). Each pair of LG-triggered and DG-triggered panels comes from a different preparation. Each row of PD spike rasters corresponds to a LG or DG burst start. While a clear pattern is visible in LG-triggered rasters, rasters appear more disordered when DG-triggered. (b) Rasters of PD spikes triggered by LG bursting (top row) or DG bursting (bottom row). In each row, time has been rescaled by the mean PD periods in that row.}
\end{figure}

\lipsum[2-4]

\begin{figure}[!htp]
\centering
\begin{fullwidth}
\includegraphics[width=\linewidth]{phase_coupling.pdf}
\end{fullwidth}
\caption{\textbf{Statistics of LG and DG burst and spike phase locking to PD} (a) Cumulative probability of LG burst starts as a function of PD phase.  (b) Cumulative probability of DG burst starts as a function of PD phase. Each curve in (a) and (b) pools all LG or DG burst starts at a given temperature. Shading indicates standard deviation estimated by bootstrapping the data 1000 times.  (c) Mean LG start in PD phase as a function of temperature. Gray lines denote individual preparations; black line is the average of all preparations. Bars indicate standard error of the mean. (d) Probability of LG spiking as a function of PD phase. (e) Probability of DG spiking as a function of PD phase.  Dashed lines in all panels indicates chance level.}
\end{figure}


\section{Discussion}




\subsection{Consequences of using calcium concentration as a proxy for cell activity}

\lipsum[5]

\begin{figure}[!htp]
\centering
\begin{fullwidth}
\includegraphics[width=\linewidth]{integer_coupling.pdf}
\end{fullwidth}
\caption{\textbf{Integer coupling between LG and PD periods is robust to temperatures up to 20\degree C} (a) LG burst periods vs. mean PD burst periods. Each dot corresponds to one LG burst period, and all the PD bursts during that LG burst period. Gray lines have integer slope. The data tends to lie along the lines with integer slope, suggesting that LG burst periods are integer multiples of PD burst periods. Color indicates temperature. (b) DG burst periods vs. PD burst periods. (c) Significand of ratio of LG (blue) or DG (orange) burst periods to PD burst period. Shading indicates standard deviation estimated by bootstrapping the data 1000 times.  Dashed line panels indicates chance level. (d) Number of PD cycles per LG (blue) or DG (red) cycle as a function of temperature. (e) Integerness of PD-LG burst coupling (blue) and PD-DG burst coupling (red) as a function of temperature. When integerness is 1, the two rhythms are exactly integer coupled, and when integerness is 0, they are randomly distributed.   }
\end{figure}


\lipsum[1]

\subsection{Subsection title}

\lipsum[2]


\begin{figure}
\begin{fullwidth}
\includegraphics[width=\linewidth]{duty_cycle.pdf}

\caption{\textbf{Duty cycles of pyloric and gastric neurons across temperature } Duty cycles of PD (top row), LG (middle) and DG (bottom row) as a function of time. Gray bars indicate stimulation times. Colors indicate temperature. Each column shows data from a different preparation.    } 
\label{fig:8}
\end{fullwidth}
\end{figure}


\subsection{Sub section title}

\lipsum[2]


\begin{figure}
\begin{fullwidth}
\includegraphics[width=\linewidth]{duty_cycle.pdf}

\caption{\textbf{Spontaneous gastric rhythms from 11C to 30C} (a) Extracellular recordings from lgn and dgn in a single preparation at six different temperatures showing LG and DG bursting. (b)  Dominant period of LG activity at different temperatures. Vertical lines separate preparations.  } 
\label{fig:9}
\end{fullwidth}
\end{figure}


\begin{figure}
\begin{fullwidth}
\includegraphics[width=\linewidth]{spont_rasters.pdf}

\caption{\textbf{Spontaneous gastric rhythms from 11C to 30C} (a) Extracellular recordings from lgn and dgn in a single preparation at six different temperatures showing LG and DG bursting. (b)  Dominant period of LG activity at different temperatures. Vertical lines separate preparations.  } 
\label{fig:9}
\end{fullwidth}
\end{figure}

\subsection{Stochastic effects can arise from low copy numbers}

\lipsum[3]

\subsection{Model predictions for non-neuronal cells}

\lipsum[2]

\subsection{Other mechanisms of compensation for growth}

\lipsum[5-6]




\section{Methods and Materials}

\subsection*{Data analysis}

\subsubsection*{Spike identification and sorting}

\subsubsection*{Measuring burst metrics}

\subsubsection*{Estimating $Q_{10}$s of burst periods}




\appendix

\end{document}
